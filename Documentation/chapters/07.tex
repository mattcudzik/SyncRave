\chapter{Podsumowanie i wnioski}
\section{Uzyskane wyniki w świetle postawionych celów i zdefiniowanych wymagań}
Aplikacja spełnia większość postawionych założeń. Prawie wszystkie funkcjonalności, które nie zostały zaimplementowane są stosunkowo proste do dodania. Przykładowo aby możliwość blokady danych gatunków muzyki działała należałoby dodać odpowiedni formularz w interfejsie użytkownika. Innym przykładem jest nadanie właścicielowi sesji możliwości usuwania gości z sesji, wymagałoby to jedynie dodania nowego punktu końcowego z prostą funkcjonalnością oraz odpowiednio podpiętego przycisku. Istnieje natomiast kilka funkcji, które aby zrealizować należałoby przeorganizować strukturę warstwy usług. Przykładem czego jest możliwość nadawania priorytetu gościowi czy mechanizm głosowania aby pomijać lub przyspieszać pojawienie się utworu. Wymaga to ingerencji w kolejkę, której serwis Spotify nie udostępnia. Możliwe jest tylko dodanie utworu lecz nie zmiana kolejności czy usunięcie z kolejki. Potencjalnym rozwiązaniem byłoby przetrzymywanie kolejki z utworami w bazie dancyh do momentu gdy powinny być odtworzone i dopiero wtedy wysyłanie ich do serwisu Spotify.

\section{Możliwe kierunki rozwoju systemu}
System można rozwinąć poprzez dodawanie kolejnych funkcjonalności, przykładowo stworzenie dedykowanej aplikacji na urządzenia mobilne. Dodanie funkcji karaoke znacznie zwiększyłoby interaktywność aplikacji, synchronizacja tekstu z dźwiękiem byłaby możliwa, należałoby jedynie uzyskać dostęp do bazy danych z tekstami utworów, które zawierają znaczniki czasowe kolejnych linijek tekstu. Dodatkowo rozszerzenie funkcjonalności o obsługę innych platform do strumieniowania muzyki umożliwiłoby większej liczbie użytkowników korzystanie z serwisu. 

Ważnym elementem jest usprawnienie interfejsu użytkownika, w szczególności dodanie responsywności na urządzeniach mobilnych, na których obecnie korzystanie z aplikacji jest niewygodne.

System mógłby także skorzystać na poprawie mechanizmów bezpieczeństwa, które nie były priorytetem w tej pracy. Obecnie aplikacje są wrażliwe na wiele rożnych rodzajów ataków ze strony osób trzecich. 

Obecnie żądania kierowane do zewnętrznego API Spotify wykonywane są w sposób synchroniczny, może to mieć duży wpływ na optymalizację w przypadku dużej ilości aktywnych użytkowników oraz powodować opóźnienia na serwerze.

\section{Problemy napotkane w trakcie pracy}
W trakcie pracy pojawiły się trudności w implementacji technicznej. Używanie szkieletów programistycznych znacznie ułatwia tworzenie oprogramowania pod warunkiem, że programista posiada dokładną wiedzę o sposobie działania narzędzi, z których korzysta. Podczas implementacji pracy zabrakło momentami tego głębszego zrozumienia sposobu funkcjonowania systemu co stanowiło przeszkodę i wymusiło konieczność pozyskania wnikliwej wiedzy na temat mechanizmów działania tych narzędzi.

Błędem okazała się niedostateczne sprawdzenie możliwości, które udostępnia serwis Spotify przez swoje API. Wynikiem czego początkowe założenia, które system powinien spełniać okazały się zbyt ciężkie do osiągnięcia w określonym czasie. Etap analizy oraz możliwości jest kluczowy podczas tworzenia projektu i nie powinien być zaniedbany.
