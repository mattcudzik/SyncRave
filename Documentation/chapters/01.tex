\chapter{Wstęp}
\label{ch:wstep}

\section{Wprowadzenie w problem/zagadnienie}
Muzyka od zawsze towarzyszyła człowiekowi. Jest obecna na każdym spotkaniu towarzyskim czy wydarzeniu grupowym i odgrywa kluczową rolę w tworzeniu panującej atmosfery. Wspólne słuchanie muzyki potrafi integrować grupę i pomaga budować relacje. 

Problematyczny może się okazać dobór odpowiedniej muzyki, która powinna angażować wszystkich uczestników i w miarę możliwości odpowiadać preferencjom jak największej liczby osób. Aplikacja, która ma za zadanie umożliwić każdemu uczestnikowi dodawania własnych ulubionych utworów do wspólnej listy, może być odpowiedzią na te wyzwania.

\section{Osadzenie problemu w dziedzinie}
Obecnie jednym z powszechnie wykorzystywanych sposobów odtwarzania muzyki są platformy do strumieniowania treści takie jak, Spotify \cite{bib:music_report}. Ręczne sterowanie odtwarzaniem lub wybór utworów przez pojedyncze osoby może nie zawsze spełnić oczekiwania wszystkich uczestników. Brak systemów, które efektywnie łączą preferencje wszystkich użytkowników oraz umożliwiają równoczesne i sprawiedliwe dodawanie utworów do wspólnej listy odtwarzania jest głównym powodem powstania niniejszej pracy.

\section{Cel pracy}
Celem tej pracy jest stworzenie aplikacji, która umożliwi spersonalizowanie zarządzania odtwarzanej muzyki podczas spotkań towarzyskich w oparciu o integrację z serwisem Spotify. System ma dać uczestnikom wydarzenia możliwość równoczesnego dodawania utworów do wspólnej kolejki odtwarzania oraz intuicyjny i wygodny interfejs do interakcji z serwisem Spotify, aby zapewnić m.in. prosty sposób przeglądania dostępnych w bazie utworów. Aplikacja umożliwi administrację sesją, pozwalając przykładowo na ustawienie blokady na treści dla dorosłych oraz kontrolę liczby uczestników. System nie wymaga rejestracji czy logowania od użytkowników, którzy do sesji dołączają, jedynym wymogiem jest znajomość kodu dostępu lub zeskanowanie kodu QR udostępnionego przez właściciela sesji. Wszystkie dodawane utwory w sesji dodawane są do wspólnej listy utworów dostępnej publicznie na koncie Spotify właściciela sesji, jeśli ta opcja nie została wyłączona.

\section{Zakres pracy oraz określenie wkładu autora}
\subsection{Baza danych}
Zaprojektowanie oraz implementacja bazy danych, dostosowanej do potrzeb aplikacji, przechowującej informacje o sesjach, gościach oraz właścicielach sesji. Ustalenie relacji między encjami w celu zapewnienia optymalnej struktury danych.
\subsection{Serwer aplikacji}
Zaprojektowanie struktury aplikacji oraz analiza wymagań. Wdrożenie logiki, na którą składa się obsługa sesji i użytkowników, komunikacja z zewnętrznym API serwisu Spotify oraz implementacja interfejsów RESTful API umożliwiających komunikację z aplikacją klienta. Zapewnienie bezpieczeństwa przez realizację mechanizmów uwierzytelnienia oraz autoryzacji dostępu do zasobów. 
\subsection{Interfejs użytkownika}
Zdefiniowanie struktury aplikacji - wyznaczenie komponentów odpowiedzialnych za różne funkcje (logowanie, zarządzanie sesją), zaplanowanie klarownej i intuicyjnej nawigacji pomiędzy widokami. Stworzenie szablonów HTML i ich stylizacja, aby poprawić interaktywność z użytkownikiem przez np. animacje sugerujące ładowanie danych. Implementacja komponentów do obsługi logiki biznesowej takich jak przyciski, panel wyszukiwania utworów czy formularz z walidacją danych.
\subsection{Komunikacja między klientem a serwerem}
Zdefiniowanie interfejsu API, określenie ścieżek URL, parametrów i metod HTTP dla poszczególnych operacji zgodnie z zasadami architektury REST. Implementacja logiki po stronie klienta odpowiedzialnej za wywoływanie odpowiednich żądań do określonych punktów końcowych serwera aplikacji oraz przetworzenie danych otrzymanych w~odpowiedzi poprzez np. aktualizację interfejsu użytkownika o odpowiednio sformatowane wyniki zapytania. Zapewnienie obsługi błędów (brak odpowiedzi ze strony serwera, niekompletne zapytanie, wewnętrzny błąd serwera itp.) oraz walidacji danych zarówno po stronie serwera jak i aplikacji klienta. 
\subsection{Dokumentacja techniczna}
Sporządzenie dokumentacji opisującej obiekty uczestniczące w komunikacji między serwerem a klientem oraz punkty końcowe serwera - przyjmowane parametry, ścieżki, zawartość zapytania, sposób autoryzacji oraz zwracane wartości. 
\section{Zwięzła charakterystyka rozdziałów}

Rozdział 1 zawiera wprowadzenie w zagadnienie oraz określenie dziedziny i zakresu pracy. Rozdział 2 to wstęp teoretyczny i porównanie istniejących rozwiązań dostępnych obecnie na rynku. W rozdziale 3 zostały określone wymagania, które system powinien spełnić oraz przypadki użycia aplikacji wraz z opisem użytych do implementacji narzędzi. Następnie w rozdziale 4 zawarto wyjaśnienie działania aplikacji od strony użytkowej, sposób instalacji oraz konfiguracji systemu. Dodatkowo przedstawione zostały przykłady oraz scenariusze korzystania z aplikacji. Kolejny rozdział skupia się na przedstawieniu sposobu działania aplikacji ze strony technicznej, przez omówienie architektury, bazy danych i implementacji systemu. Rozdział 6 to omówienie metod i wyników weryfikacji oraz walidacji systemu. Finalnie w rozdziale 7 przedstawione zostało podsumowanie pracy oraz zdefiniowanie przyszłych kierunków rozwoju systemu.

